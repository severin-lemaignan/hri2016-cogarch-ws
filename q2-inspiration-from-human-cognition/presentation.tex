%!TEX program = xelatex

\documentclass[compress]{beamer}
%--------------------------------------------------------------------------
% Common packages
%--------------------------------------------------------------------------
\usepackage[english]{babel}
\usepackage{pgfpages} % required for notes on second screen
\usepackage{graphicx}

\usepackage{multicol}

\usepackage{tabularx,ragged2e}
\usepackage{booktabs}


%--------------------------------------------------------------------------
% Load theme
%--------------------------------------------------------------------------
\usetheme{hri}

\usepackage{dtklogos} % must be loaded after theme
\usepackage{tikz}
\usetikzlibrary{mindmap,backgrounds,positioning}

\graphicspath{{figs/}}
\setbeamercolor{refToContribCol}{bg=hriSec1Comp,fg=white}
\newcommand{\refToContrib}[1]{%
    \begin{beamercolorbox}[wd=\linewidth,ht=2ex,dp=0.7ex]{refToContribCol}%
    \hspace{0.5em}$\hookrightarrow$ #1%
    \end{beamercolorbox}%
}%

%--------------------------------------------------------------------------
% General presentation settings
%--------------------------------------------------------------------------
\title{Models of Human Cognition}
\subtitle{Q2 -- Should cognitive architectures for social interaction be
inspired and/or limited by models of human cognition?}
\date{7th March 2016\\ {\tiny \url{https://sites.google.com/site/cogarch4socialhri2016/}}}
\author{}
\institute{\includegraphics[height=15mm]{plymouth-logo}}
%\\Centre for Robotics \& Neural
%Systems\\{\Medium Plymouth University}}

%--------------------------------------------------------------------------
% Notes settings
%--------------------------------------------------------------------------
%\setbeameroption{show notes on second screen}

\begin{document}
%--------------------------------------------------------------------------
% Titlepage
%--------------------------------------------------------------------------

\maketitle

%\imageframe[Children playing with\\the Ranger robot]{photo-fullscreen.jpg}
%\imageframe{photo-fullscreen.jpg}


\begin{frame}{}

	Implicit assumption that it is the robot that should conform to the human...
	
	\begin{itemize}
        \item Even though available evidence suggests that the converse will happen anyway: The human as the reference point
    \end{itemize}
    
    Consequences:
    
    \begin{itemize}
	
	\item<2-> Overlap of perception/action competencies between human and robot 
	
	\item<3-> Predictability of robot behaviour to humans 
	
	\item<4-> Time-scales of the robot should be the same as that of the human 
	
	\end{itemize}
	
	\refToContrib{At least implicit in all contributions...}

\end{frame}


\begin{frame}{Role of human cognition?}

	\begin{multicols}{2}
        \resizebox{0.7\columnwidth}{!}{%
            \begin{tikzpicture}[
                    >=latex,
                every edge/.style={<-, draw, very thick}]

            \path[small mindmap, 
                level 1 concept/.append style={sibling angle=360/6}, 
                level 2 concept/.append style={sibling angle=60}, 
            concept color=hriSec1,text=white]
            node[concept] {Human Cognition?}
                [clockwise from=60]
                child[concept color=hriSec3Dark,text=white] { node[concept]{1. Inspiration} }
                child[concept color=hriSec2Dark,text=white] { node[concept]{2. Limitations} }
                child[concept color=hriSec2CompDark,text=white] { node[concept] {3. Based on}};


        \end{tikzpicture}
    }
    
    \vspace{0.5cm}
    
	\footnotesize{Multiple levels of human cognitive models \cite{Sun2005} }
    
    \vfill
    \columnbreak

    \uncover<2->{ {\Medium 1. Inspiration from Human Cognition}

    {\scriptsize -- Modelling aspects of human cognition \\-- Subsequent application to robots} }

    \uncover<3->{ {\Medium 2. Take both Inspiration and Constraints from Human Cognition}

    {\scriptsize -- Examine human cognition limitations \\-- Constrain competencies of the robot} }

    \uncover<4->{ {\Medium 3. Base entirely on Human Cognition}

    {\scriptsize -- Functions and architecture structure \\-- Typically neural-inspired implementation} }

    \end{multicols}
    
    \refToContrib{All contributions address one or more of these points}

\end{frame}


\begin{frame}{}

    \begin{exampleblock}{No suggestion here that any of the three approaches are ``\textit{better}'' in any way!}
        
    \end{exampleblock}
    \vspace{2em}
    \centering
    \uncover<2>{
        \small{ \textit{Although maybe this should be a point for discussion...} }
    }

\end{frame}


\begin{frame}{1. Inspiration from Human Cognition}

	\begin{itemize}
        \item From the `top-down': start with `model' of human social behaviour (\textit{surface level}), apply to robot in terms of behaviour
        \item Of greater importance is the human-like behaviour - mechanism of lesser importance
        \item E.g.: requirements for robot social skills \cite{Dautenhahn2007}:\\-- contact, functionality, role, and social skills
    \end{itemize}
    
    \refToContrib{Papers by Ignacio, Liz...}

\end{frame}


\begin{frame}{2. Also take Constraints from Human Cognition}

	\begin{itemize}
        \item Instead of trying to maximise performance of the robot system,
            intentionally inhibit performance to human-levels \\ {\footnotesize -- humans are definitely sub-optimal }
        \item Can feed into the matching of human expectations to facilitate predictability
        \item E.g.: modifying robot reaching to be more legible \cite{Dragan2013}, or label learning through embodiment \cite{Morse2010}
        \item<2-> Possibility that insights from human cognition can lead to unexpected advantages: e.g. integration of `emotional' processing \cite{Gros2010}
    \end{itemize}
    
    \refToContrib{Papers by Ignacio, Nick, Paul, ...}

\end{frame}


\begin{frame}{3. Base entirely on Human Cognition}

	\begin{itemize}
		\item From the `bottom-up': create a model of the human cognitive system, and subsequently apply to HRI
		\item Typically neural-based: neural networks at various levels of abstraction
        \item E.g.: large-scale structured and embodied spiking neural networks \cite{Krichmar2002} \footnotesize{(not HRI!)}
    \end{itemize}
    
    \refToContrib{Papers by Vasanth (DIRAC arch), ...}

\end{frame}


\begin{frame}{Discussion points}
    
	-- Since humans are highly adaptive anyway, should this be explicitly taken into account in the architecture, and how?
	\refToContrib{Related to ToM, papers by Sandra, Liz \& Alexis}

	-- Along the same lines, many of the social interactions are actually based
	on social conventions: should these be encoded in the architecture as well?
    \refToContrib{Vasanth's contribution}
	
	-- Does the robot being predictable for human interactants imply the robot should behave (and/or \textit{cognate}) like humans?
	\refToContrib{Paper by Gabriel vs Vasanth, ...}
	
	-- To achieve long-term social HRI, what are the relative benefits/shortfalls of the three general approaches?
	\refToContrib{general...}
	
	\vspace{0.5cm}

    35 minutes of open discussion, plus one presentation.

\end{frame}



\begin{frame}{Bibliography}
\begin{thebibliography}{10}

	\tiny{

    \beamertemplatearticlebibitems
    \bibitem{Sun2005}
    R. Sun
    \newblock ``On levels of cognitive modeling''
    \newblock 2005

    \beamertemplatearticlebibitems
    \bibitem{Krichmar2002}
    J. Krichmar, G. Edelman
    \newblock ``Machine Psychology: Autonomous behavior, perceptual categorisation and conditioning in a brain-based device''
    \newblock 2002

    \beamertemplatearticlebibitems
    \bibitem{Dautenhahn2007}
    K. Dautenhahn
    \newblock ``Socially intelligent robots: dimensions of human-robot interaction''
    \newblock 2007

    \beamertemplatearticlebibitems
    \bibitem{Dragan2013}
    A. Dragan, K. Lee, S. Srinivasa
    \newblock ``Legibility and predictability of robot motion''
    \newblock 2013    
    
	\beamertemplatearticlebibitems
    \bibitem{Gros2010}
    C. Gros
    \newblock ``Cognition and Emotion: Perspectives of a Closing Gap''
    \newblock 2010
    
    \beamertemplatearticlebibitems
    \bibitem{Morse2010}
    A. Morse, \textit{et al}
    \newblock ``Epigenetic Robotics Architecture (ERA)''
    \newblock 2010
    
    }


  \end{thebibliography}
\end{frame}

\end{document}







%!TEX program = xelatex

\documentclass[compress]{beamer}
%--------------------------------------------------------------------------
% Common packages
%--------------------------------------------------------------------------
\usepackage[english]{babel}
\usepackage{pgfpages} % required for notes on second screen
\usepackage{graphicx}

\usepackage{multicol}

\usepackage{tabularx,ragged2e}
\usepackage{booktabs}


%--------------------------------------------------------------------------
% Load theme
%--------------------------------------------------------------------------
\usetheme{hri}

\usepackage{dtklogos} % must be loaded after theme
\usepackage{tikz}
\usetikzlibrary{mindmap,backgrounds,positioning}

\graphicspath{{figs/}}

%--------------------------------------------------------------------------
% General presentation settings
%--------------------------------------------------------------------------
\title{Functional Requirements}
\subtitle{Q3 -- What are the functional requirements for a cognitive
architecture to support social interaction?}
\date{}
\author{}
\institute{\includegraphics[height=15mm]{plymouth-logo}}
%\\Centre for Robotics \& Neural
%Systems\\{\Medium Plymouth University}}

%--------------------------------------------------------------------------
% Notes settings
%--------------------------------------------------------------------------
%\setbeameroption{show notes on second screen}

\begin{document}
%--------------------------------------------------------------------------
% Titlepage
%--------------------------------------------------------------------------

\maketitle

%\imageframe[Children playing with\\the Ranger robot]{photo-fullscreen.jpg}
\imageframe{photo-fullscreen.jpg}

\begin{frame}{Cognitive Architecture?}
\centering
        \resizebox{!}{0.7\paperheight}{%
            \begin{tikzpicture}[
                    >=latex,
                every edge/.style={<-, draw, very thick}]
        

            \path[small mindmap, 
                level 1 concept/.append style={sibling angle=360/5}, 
                level 2 concept/.append style={sibling angle=120}, 
            concept color=hriWarmGreyLight,text=hriWarmGreyDark]
            node[concept] {Cognition}
            [clockwise from=-30]
            child[concept color=hriSec1,text=white] { node[concept] (percept) {Perception}
                [clockwise from=40]
                child[concept color=hriSec2Dark,text=white] { node[concept]{Attention} }
                child[concept color=hriSec2CompDark,text=white] { node[concept] (dialog) {Communication} }
            }
            child[concept color=hriSec2Comp,text=white] { node[concept] (krr) {Knowledge Manipulation} 
                [clockwise from=-30]
                child[concept color=hriSec1CompDark,text=white] { node[concept] (memory) {Memory} }
                child[concept color=hriSec3CompDark,text=white] { node[concept] (tom) {Theory of Mind} }
            }
            child[concept color=hriSec2,text=white] { node[concept] (action) {Action Execution} 
                [counterclockwise from=110]
                child[concept color=hriSec3,text=white] { node[concept] (plan) {Task Planning} }
                child[concept color=hriSec2CompDark,text=white] { node[concept] (comm) {Communication} }
            }
            child[concept color=hriSec3Comp,text=white] { node[concept] (learning) {Learning} 
                [counterclockwise from=130]
                child[concept color=hriSec1CompDark,text=white] { node[concept] (adapt) {Adaptation} }
            }
            child[concept color=hriSec3CompDark,text=white] { node[concept] (reason) {Reasoning} 
                [clockwise from=110]
                child[concept color=hriSec3Dark,text=white] { node[concept] {Problem solving} }
                child[concept color=hriSec1Dark,text=white] { node[concept] (decision) {Decision making} } 
            };

        \onslide<2>{
        \path (decision) edge[->, bend left] (action);
        \path (percept) edge[->, bend left] (action);
        \path (krr) edge[<->, bend right] (reason);
        \path (plan) edge[<->] (reason);
        \path (percept) edge[->, bend left] (krr);
        \path (krr) edge[->, bend left] (plan);
        \path (learning) edge[->, bend left] (reason);
        \path (learning) edge[->, bend left] (krr);
        \path (percept) edge[<->, bend left] (learning);
        \path (dialog) edge[->, bend left] (tom);
        \path (tom) edge[->, bend left] (comm);
        \path (percept) edge[<->, bend left] (memory);
        }


        \end{tikzpicture}
    }
\end{frame}

\begin{frame}{}

    \centering

    \begin{exampleblock}{Cognitive Architectures Building $\neq$ Science of Integration}
        The function may be a {\Medium byproduct} of the architecture
    \end{exampleblock}
    \vspace{2em}
    \uncover<2>{
        At this stage, we should not commit to any architecture,\\nor internal mechanism
    }


\end{frame}


\begin{frame}{No commitments to internal mechanisms}
    \centering
        \resizebox{!}{0.7\paperheight}{%
            \begin{tikzpicture}[
                    >=latex,
                every edge/.style={<-, draw, very thick}]
        
         \onslide<1>{
            \path[small mindmap, 
                level 1 concept/.append style={sibling angle=360/5}, 
                level 2 concept/.append style={sibling angle=120}, 
            concept color=hriWarmGreyLight,text=hriWarmGreyDark]
            node[concept color=white] {}
            [clockwise from=-30]
            child[concept color=hriSec1,text=white] { node[concept] (percept) {Perception}
                [clockwise from=40]
                child[concept color=hriSec2Dark,text=white] { node[concept]{Attention}; }
                child[concept color=hriSec2CompDark,text=white] { node[concept] (dialog) {Communication} ;};
            }
            child[concept color=hriSec2Comp,text=white] { node[concept] (krr) {Knowledge Manipulation} 
                [clockwise from=-30]
                child[concept color=hriSec1CompDark,text=white] { node[concept] (memory) {Memory}; }
                child[concept color=hriSec3CompDark,text=white] { node[concept] (tom) {Theory of Mind}; };
            }
            child[concept color=hriSec2,text=white] { node[concept] (action) {Action Execution} 
                [counterclockwise from=110]
                child[concept color=hriSec3,text=white] { node[concept] (plan) {Task Planning}; }
                child[concept color=hriSec2CompDark,text=white] { node[concept] (comm) {Communication}; };
            }
            child[concept color=hriSec3Comp,text=white] { node[concept] (learning) {Learning} 
                [counterclockwise from=130]
                child[concept color=hriSec1CompDark,text=white] { node[concept] (adapt) {Adaptation}; };
            }
            child[concept color=hriSec3CompDark,text=white] { node[concept] (reason) {Reasoning} 
                [clockwise from=110]
                child[concept color=hriSec3Dark,text=white] { node[concept] {Problem solving};}
                child[concept color=hriSec1Dark,text=white] { node[concept] (decision) {Decision making}; } ;
            };
        }
         \onslide<2>{
            \path[small mindmap, 
                level 1 concept/.append style={sibling angle=360/5}, 
                level 2 concept/.append style={sibling angle=120}, 
            concept color=hriWarmGreyLight,text=hriWarmGreyDark]
            node[concept color=white] {}
            [clockwise from=-30]
            child[concept color=hriSec1,text=white] { node[concept] (percept) {Perception}
                [clockwise from=40]
                child[concept color=hriSec2Dark,text=white] { node[concept]{Attention}; }
                child[concept color=hriSec2CompDark,text=white] { node[concept] (dialog) {Communication} ;};
            }
            child[concept color=white,text=gray] { node[concept] (krr) {Knowledge Manipulation} 
                [clockwise from=-30]
                child[concept color=white,text=gray] { node[concept] (memory) {Memory}; }
                child[concept color=hriSec3CompDark,text=white] { node[concept] (tom) {Theory of Mind}; };
            }
            child[concept color=hriSec2,text=white] { node[concept] (action) {Action Execution} 
                [counterclockwise from=110]
                child[concept color=white,text=gray] { node[concept] (plan) {Task Planning}; }
                child[concept color=hriSec2CompDark,text=white] { node[concept] (comm) {Communication}; };
            }
            child[concept color=white,text=gray] { node[concept] (learning) {Learning} 
                [counterclockwise from=130]
                child[concept color=hriSec1CompDark,text=white] { node[concept] (adapt) {Adaptation}; };
            }
            child[concept color=white,text=gray] { node[concept] (reason) {Reasoning} 
                [clockwise from=110]
                child[concept color=white,text=gray] { node[concept] {Problem solving};}
                child[concept color=white,text=gray] { node[concept] (decision) {Decision making}; } ;
            };
        }

        \end{tikzpicture}
    }


\end{frame}

\begin{frame}{}

    \centering

    \begin{exampleblock}{Cognitive Architectures Building $\neq$ Science of Integration}
        The function may be a {\Medium byproduct} of the architecture
    \end{exampleblock}
    \vspace{2em}
    What are the {\Medium surface} cognitive competencies\\required by {\Medium social interaction}?


\end{frame}



\begin{frame}{Functions for Social Cognition}
\centering
        \resizebox{!}{0.7\paperheight}{%
            \begin{tikzpicture}[
                    >=latex,
                every edge/.style={<-, draw, very thick}]
        

            \path[small mindmap, 
                level 1 concept/.append style={sibling angle=360/6}, 
                level 2 concept/.append style={sibling angle=60}, 
            concept color=hriWarmGreyLight,text=hriWarmGreyDark]
            node[concept] {Social\\Cognition in HRI}
            [clockwise from=30]
            child[concept color=hriSec1,text=white] { node[concept] (percept) {Perception of Human's State}
                [clockwise from=120]
                child[concept color=hriSec3Dark,text=white] { node[concept]{Emotions} }
                child[concept color=hriSec2Dark,text=white] { node[concept]{Attention} }
                child[concept color=hriSec2CompDark,text=white] { node[concept] {Inference of mental models} }
            }
            child[concept color=hriSec2Comp,text=white,grow=-30] { node[concept] (knowledge) {Social Knowledge} 
                [counterclockwise from=-120]
                child[concept color=hriSec1CompDark,text=white] { node[concept] {Social rules} }
                child[concept color=hriSec3Comp,text=black] { node[concept] {Social context} }
                child[concept color=hriSec3CompDark,text=white] { node[concept] {Common-sense} }
            }
            child[concept color=hriSec3Comp,text=black, grow=-120] { node[concept] (comm) {Communication} 
                [counterclockwise from=180]
                child[concept color=hriSec1CompDark,text=white] { node[concept] {Verbal} }
                child[concept color=hriSec1CompDark,text=white] { node[concept] {Non-verbal} }
            }
            child[concept color=hriSec3,text=white,grow=180] { node[concept] (dynamics) {Interaction Dynamics} 
                [clockwise from=180]
                child[concept color=hriSec2Dark,text=white] { node[concept] {Long-term interaction} }
            }
            child[concept color=hriSec2,text=black, grow=120] { node[concept] (action) {Human-aware actions} 
                [counterclockwise from=120]
                child[concept color=hriSec2CompDark,text=white] { node[concept] {Action, behaviour recognition} }
                child[concept color=hriSec3,text=white] { node[concept] {Joint actions} }
            };


        \end{tikzpicture}
    }
%    \\
%    Learn -- Memorize -- Reason -- Represent -- Assess Situation
\end{frame}


\begin{frame}{Perception of human's cognitive state}

    \begin{multicols}{2}
        \resizebox{0.7\columnwidth}{!}{%
            \begin{tikzpicture}[
                    >=latex,
                every edge/.style={<-, draw, very thick}]

            \path[small mindmap, 
                level 1 concept/.append style={sibling angle=360/6}, 
                level 2 concept/.append style={sibling angle=60}, 
            concept color=hriSec1,text=white]
            node[concept] {Perception of Human's State}
                [clockwise from=60]
                child[concept color=hriSec3Dark,text=white] { node[concept]{Emotions} }
                child[concept color=hriSec2Dark,text=white] { node[concept]{Attention} }
                child[concept color=hriSec2CompDark,text=white] { node[concept] {Inference of mental models}};


        \end{tikzpicture}
    }
    \vfill
    \columnbreak

    {\Medium Emotions}

    Facial features [CITE?]; non-verbal acoustic features [CITE?]; verbal features (sentient analysis)

    {\Medium Attention}

    Visual attention

    {\Medium Mental Modelling}

    User modelling; (perceptual) Theory of Mind~\cite{lemaignan2015mutual};

    \end{multicols}
\end{frame}

\begin{frame}{Social Knowledge}

    \begin{multicols}{2}
        \resizebox{0.7\columnwidth}{!}{%
            \begin{tikzpicture}[
                    >=latex,
                every edge/.style={<-, draw, very thick}]

            \path[small mindmap, 
                level 1 concept/.append style={sibling angle=360/6}, 
                level 2 concept/.append style={sibling angle=60}, 
            concept color=hriSec2Comp,text=white]
            node[concept] (knowledge) {Social Knowledge} 
                [counterclockwise from=-60]
                child[concept color=hriSec1CompDark,text=white] { node[concept] {Social rules} }
                child[concept color=hriSec3Comp,text=black] { node[concept] {Social context} }
                child[concept color=hriSec3CompDark,text=white] { node[concept] {Common-sense}};


        \end{tikzpicture}
    }
    \vfill
    \columnbreak

    {\Medium Common-sense}

    Web-scrapping~\cite{tenorth2010understanding}

    {\Medium Social context}

    ?

    {\Medium Social Rules}

    Proxemics; ?

    \end{multicols}
\end{frame}

\begin{frame}{Communication}

    \begin{multicols}{2}
        \resizebox{0.7\columnwidth}{!}{%
            \begin{tikzpicture}[
                    >=latex,
                every edge/.style={<-, draw, very thick}]

            \path[small mindmap, 
                level 1 concept/.append style={sibling angle=360/6}, 
                level 2 concept/.append style={sibling angle=60}, 
            concept color=hriSec3Comp,text=black]
            node[concept] (comm) {Communication} 
                [counterclockwise from=-30]
                child[concept color=hriSec1CompDark,text=white] { node[concept] {Verbal} }
                child[concept color=hriSec1CompDark,text=white] { node[concept] {Non-verbal} };

        \end{tikzpicture}
    }
    \vfill
    \columnbreak


    Review~\cite{mavridis2015review}:
    \begin{quote}
        Breaking the ``simple commands only'' barrier;\\
        Mixed initiative dialogue;\\
        Situated language and the symbol grounding problem;\\
        Motor correlates and Non-Verbal Communication...
    \end{quote}


    {\Medium Non-verbal}

    Behavioural alignment

    {\Medium Verbal}

    Situated dialogue~\cite{kruijff2010situated}

    \end{multicols}

\end{frame}

\begin{frame}{Interaction Dynamics}

    \begin{multicols}{2}
        \resizebox{0.7\columnwidth}{!}{%
            \begin{tikzpicture}[
                    >=latex,
                every edge/.style={<-, draw, very thick}]

            \path[small mindmap, 
                level 1 concept/.append style={sibling angle=360/6}, 
                level 2 concept/.append style={sibling angle=60}, 
            concept color=hriSec3,text=white]
            node[concept] (dynamics) {Interaction Dynamics} 
                [clockwise from=0]
                child[concept color=hriSec2Dark,text=white] { node[concept] {Long-term interaction} };

        \end{tikzpicture}
    }
    \vfill
    \columnbreak




    Behavioural alignment

    {\Medium Long-term interaction}

    Situated dialogue~\cite{kruijff2010situated}

    \end{multicols}

\end{frame}


\begin{frame}{Human-aware actions}

    \begin{multicols}{2}
        \resizebox{0.7\columnwidth}{!}{%
            \begin{tikzpicture}[
                    >=latex,
                every edge/.style={<-, draw, very thick}]

            \path[small mindmap, 
                level 1 concept/.append style={sibling angle=360/6}, 
                level 2 concept/.append style={sibling angle=60}, 
            concept color=hriSec2,text=black]
            node[concept] (action) {Human-aware actions} 
                [counterclockwise from=-30]
                child[concept color=hriSec2CompDark,text=white] { node[concept] {Action, behaviour recognition} }
                child[concept color=hriSec3,text=white] { node[concept] {Joint actions} };
        \end{tikzpicture}
    }
    \vfill
    \columnbreak




    {\Medium Action \& Behaviour recognition}

    {\Medium Joint action}


    \end{multicols}

\end{frame}

\begin{frame}{Discussion directions}
    \begin{itemize}
        \item Is it the right approach?
        \item Missing functional requirements for social interaction?
        \item Relations between functions? Functional entanglement?
        \item Evaluations? Tests? Benchmarks? At what level are they useful? (single function? group of functions? whole architecture only?)
    \end{itemize}
\end{frame}


\begin{frame}{Bibliography}
\begin{thebibliography}{10}

    \beamertemplatebookbibitems
    \bibitem{Oppenheim2009}
    Alan~V.~Oppenheim
    \newblock \doublequoted{Discrete-Time Signal Processing}
    \newblock Prentice Hall Press, 2009

    \beamertemplatearticlebibitems
    \bibitem{lemaignan2015mutual}
    Séverin Lemaignan, Pierre Dillenbourg
    \newblock \doublequoted{Mutual Modelling: Inspiration for the Next Steps}
    \newblock 2015

    \beamertemplatearticlebibitems
    \bibitem{tenorth2010understanding}
    Moritz Tenorth, Daniel Nyga, Michael Beetz
    \newblock \doublequoted{Understanding and Executing Instructions for Everyday Manipulation Tasks
    from the World Wide Web}
    \newblock 2010

    \beamertemplatearticlebibitems
    \bibitem{mavridis2015review}
    N. Mavridis
    \newblock \doublequoted{A review of verbal and non-verbal human–robot interactive communication}
    \newblock 2015



    \beamertemplatearticlebibitems
    \bibitem{kruijff2010situated}
    G.-J. M. Kruijff, P. Lison, T. Benjamin, H. Jacobsson, H. Zender, I. Kruijff-Korbayová, N. Hawes
    \newblock \doublequoted{Situated dialogue processing for human-robot interaction}
    \newblock 2010



  \end{thebibliography}
\end{frame}

\end{document}







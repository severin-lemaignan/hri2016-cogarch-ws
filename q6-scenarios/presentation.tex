%!TEX program = xelatex

\documentclass[compress]{beamer}
%--------------------------------------------------------------------------
% Common packages
%--------------------------------------------------------------------------
\usepackage[english]{babel}
\usepackage{pgfpages} % required for notes on second screen
\usepackage{graphicx}

\usepackage{multicol}

\usepackage{tabularx,ragged2e}
\usepackage{booktabs}


%--------------------------------------------------------------------------
% Load theme
%--------------------------------------------------------------------------
\usetheme{hri}

\usepackage{dtklogos} % must be loaded after theme
\usepackage{tikz}
\usetikzlibrary{mindmap,backgrounds,positioning}

\graphicspath{{figs/}}

\setbeamercolor{refToContribCol}{bg=hriSec1Comp,fg=white}
\setbeamercolor{highlightCol}{bg=hriSec3,fg=white}

\newcommand{\refToContrib}[1]{%
    \begin{beamercolorbox}[wd=\linewidth,ht=2ex,dp=0.7ex]{refToContribCol}%
    \hspace{0.5em}$\hookrightarrow$ #1%
    \end{beamercolorbox}%
}%

\newcommand{\highlight}[1]{%
    \begin{beamercolorbox}[wd=\linewidth,dp=0.7ex]{highlightCol}%
    #1%
    \end{beamercolorbox}%
}%

%--------------------------------------------------------------------------
% General presentation settings
%--------------------------------------------------------------------------
\title{Soc. Interaction Scenarios}
\subtitle{Q6 -- Can you devise a social interaction scenario that current
cognitive architectures would likely fail, and why?}
\date{7th March 2016\\ {\tiny \url{https://sites.google.com/site/cogarch4socialhri2016/}}}
\author{}
\institute{\includegraphics[height=15mm]{plymouth-logo}}
%\\Centre for Robotics \& Neural
%Systems\\{\Medium Plymouth University}}

%--------------------------------------------------------------------------
% Notes settings
%--------------------------------------------------------------------------
%\setbeameroption{show notes on second screen}

\begin{document}

\maketitle

\begin{frame}{Why?}
    Why would we want to work out scenarios?

    \begin{itemize}
        \item {\Medium evidence the need} for (socio-)cognitive architectures in
            robotics
        \item {\Medium build up a research agenda}
        \item {\Medium measure the progress}
    \end{itemize}

    \uncover<2>{
        And, as it turns out, it is rather hard to come up with compelling interaction
        scenarios...
    }

\end{frame}

\begin{frame}{Why?}

    Less far-reaching objective:
    \vspace{1em}

    \highlight{\centering Interaction scenarios to {\Medium push (and break!)
    our current architectures}, and {\Medium inform} us on the choice of computational models}

\end{frame}

\begin{frame}{}

    The promise of socio-cognitive architectures is to enable our
    robots to {\Medium deal with complex (multi-modal, 4D), human-like, partially unpredictable
    situations}.

    \vspace{2em}

    \uncover<2>{
        We want to challenge that :-)
    }

\end{frame}


\begin{frame}{}

    {\Medium scenario? task? benchmark?}

    \begin{itemize}
        \item<2-> {\Medium toy scenarios}: do not say much about real interactions
        \item<3-> {\Medium ``in the wild''}: rich \& certainly necessary, but also hard to replicate (either
            locally or in other labs) and time consuming
        \item<4-> {\Medium ``unit-tasks''} (benchmarks): risk of developing/tuning a system for
            a single purpose -- contradict the mere idea of cognitive architectures
        \item<5-> The scenario {\Medium tells a story}: it projects into an
            usage\\ $\Rightarrow$ resulting ``cognitive noise'' may or may not be desirable
    \end{itemize}
\end{frame}


\begin{frame}{Desirable meta-features}

    \begin{itemize}
        \item easy to reproduce in different locations (location, culture,
            platform agnostic)
        \item grounded in existing literature (socio-cognitive psychology, developmental
            psychology, neurosciences, etc.)
        \item should exhibit all the features of a good experimental protocol,
            including a clear baseline (typically a human baseline)
        \item hard enough to be challenging!
    \end{itemize}
\end{frame}

\begin{frame}{Framing}
    The (set of) tasks must be {\Medium carefully framed}:\\

    They should {\Medium only necessitate the cognitive skills that they
    attempt to evidence}.

\end{frame}

\begin{frame}{Check-list for a good interaction scenario}
    \scriptsize
    \begin{multicols}{2}

        {\Medium Nature of the task}\\
        $\square$ requires social interaction\\
        $\square$ mirrors/models natural interactions\\
        $\square$ tests more than one cognitive function\\


        {\Medium Experimental protocol}\\
        $\square$ based on existing, published protocol\\
        Which one: \line(1,0){80}\\
        $\square$ well framed: actual cognitive requirements match what is tested\\
        $\square$ has clear experimental conditions\\
        $\square$ has a meaningful baseline\\

        {\Medium Reproducibility}\\
        $\square$ meaningful for a range of robots\\
        $\square$ does not require an overspecific participant population\\
        $\square$ readily reproducible task environment\\

        {\Medium Impact}\\
        $\square$ is socially relevant\\
        $\square$ the intent of the task can be explained in one sentence\\
        $\square$ it is beyond the state-of-art\\

        {\Medium Required socio-cognitive functions}

        \begin{multicols}{2}
            $\square$ Perception\\
            {\tiny
                \hspace*{0.4cm}$\square$ emotions\\
                \hspace*{0.4cm}$\square$ attention\\
                \hspace*{0.4cm}$\square$ mentalization\\
            }
            $\square$ Social knowledge\\
            {\tiny
                \hspace*{0.4cm}$\square$ common-sense\\
                \hspace*{0.4cm}$\square$ social ctxt\\
                \hspace*{0.4cm}$\square$ social rules\\
            }
            \vfill
            \columnbreak
            $\square$ Communication\\
            {\tiny
                \hspace*{0.4cm}$\square$ verbal\\
                \hspace*{0.4cm}$\square$ non-verbal\\
            }
            $\square$ Dynamics\\
            {\tiny
                \hspace*{0.2cm}$\square$ long-term interaction\\
            }
            $\square$ Actions\\
            {\tiny
                \hspace*{0.4cm}$\square$ action recognition\\
                \hspace*{0.4cm}$\square$ joint actions\\
            }
        \end{multicols}




    \end{multicols}

\end{frame}


\begin{frame}{A few examples}
    \small
    \begin{multicols}{2}
    \begin{itemize}
        \item<1-> RobotCup@Home (shopping)
        \item<2-> TUM's James \& Rosie cooking \vfill\columnbreak
        \item<3-> KeJia's waiter robot
        \item<4-> The ``Theory of Mind'' Task Zoo
    \end{itemize}
    \end{multicols}

    \centering
    \only<1>{
        \video{0.7\linewidth}{videos/Robocup@Home.mp4?start=241}
    }
    \only<2>{
        \video{0.7\linewidth}{videos/JamesAndRosie.mp4?start=55}
    }
    \only<3>{
        \video{0.7\linewidth}{videos/KejiaRobot.mov?start=70}
    }
    \uncover<4>{
            \includegraphics[width=0.7\linewidth]{zoo}
    }
\end{frame}

\begin{frame}{Discussion directions}

    \begin{itemize}
        \item Should we build an experimental agenda for robotic social cognition?
        \item What is the ideal ``cognitive breadth'' for a good scenario? (how many
            different cognitive skills are required?)
        \item What are these scenarios!?
    \end{itemize}

    35 minutes of open discussion, plus two presentations

\end{frame}

%\begin{frame}{Bibliography}
%\begin{thebibliography}{10}
%
%    \beamertemplatearticlebibitems
%    \bibitem{lemaignan2015mutual}
%    S. Lemaignan, P. Dillenbourg
%    \newblock \doublequoted{Mutual Modelling: Inspiration for the Next Steps}
%    \newblock 2015
%
%    \beamertemplatearticlebibitems
%    \bibitem{tenorth2010understanding}
%    M. Tenorth, D. Nyga, M. Beetz
%    \newblock \doublequoted{Understanding and Executing Instructions for Everyday Manipulation Tasks
%    from the World Wide Web}
%    \newblock 2010
%
%    \beamertemplatearticlebibitems
%    \bibitem{mavridis2015review}
%    N. Mavridis
%    \newblock \doublequoted{A review of verbal and non-verbal human–robot interactive communication}
%    \newblock 2015
%
%
%
%    \beamertemplatearticlebibitems
%    \bibitem{kruijff2010situated}
%    G.-J. M. Kruijff \emph{et al.}
%    \newblock \doublequoted{Situated dialogue processing for human-robot interaction}
%    \newblock 2010
%
%
%
%  \end{thebibliography}
%\end{frame}

\end{document}







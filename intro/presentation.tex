%!TEX program = xelatex

\documentclass[compress]{beamer}
%--------------------------------------------------------------------------
% Common packages
%--------------------------------------------------------------------------
\usepackage[english]{babel}
\usepackage{pgfpages} % required for notes on second screen
\usepackage{graphicx}

\usepackage{multicol}
\usepackage{url}

\usepackage{tabularx,ragged2e}
\usepackage{booktabs}


\usetheme{hri}

\usepackage{dtklogos} % must be loaded after theme
\usepackage{tikz}
\usetikzlibrary{calc,mindmap,backgrounds,positioning}

\graphicspath{{figs/}}
\setbeamercolor{refToContribCol}{bg=hriSec1Comp,fg=white}
\newcommand{\refToContrib}[1]{%
    \begin{beamercolorbox}[wd=\linewidth,ht=2ex,dp=0.7ex]{refToContribCol}%
    \hspace{0.5em}$\hookrightarrow$ #1%
    \end{beamercolorbox}%
}%

\title{CogArch for Social HRI}
\subtitle{2nd Workshop on Cognitive Architectures for Social Human-Robot
Interaction}
\date{7th March 2016\\ {\tiny \url{https://sites.google.com/site/cogarch4socialhri2016/}}}
\author{\scriptsize Paul Baxter, Séverin Lemaignan, Greg Trafton}
\institute{\includegraphics[height=15mm]{plymouth-logo}}
%\\Centre for Robotics \& Neural
%Systems\\{\Medium Plymouth University}}

%--------------------------------------------------------------------------
% Notes settings
%--------------------------------------------------------------------------
%\setbeameroption{show notes on second screen}

\begin{document}
%--------------------------------------------------------------------------
% Titlepage
%--------------------------------------------------------------------------

\maketitle

%\imageframe[black]{}{islands1}
%\imageframe[black]{}{islands2}
%\imageframe[black]{}{islands3}
%\imageframe[black]{}{islands4}


\begin{frame}{Context}

	\begin{itemize}
	
	\item<1-> This is the Second Workshop on Cognitive Architectures for HRI \\ {\footnotesize -- First one held in Bielefeld at HRI'14}
	
	\item<2-> First incarnation focused generally on Cognitive Architectures \\ {\footnotesize -- Keynotes from three differing approaches \\-- Somewhat lacking in discussion time!}
	
	\item<3-> This 2nd version focusses specifically on CogArch for \emph{Social} HRI \\ {\footnotesize -- Structure of this workshop intended to facilitate this exploration \\-- Plenty of time for discussion!}
	
	\end{itemize}

\end{frame}


\begin{frame}{Motivation}

	\begin{itemize}
	
	\item<1-> CogArch's seek to account for cognition using domain-general structures, mechanisms and processes \\ {\footnotesize -- Perspective from CogSci / Psychology \\-- Emphasis on domain generality, though naturally derived from domain-specific observations}
	
	\item<2-> Social Interaction as a highly complex problem \\ {\footnotesize -- Multiple facets to be taken into account \\-- Not fully characterised in humans}
	
	\item<3-> CogArch's as a means of drawing the available human-based evidence into HRI \\ {\footnotesize -- Formalising observations of phenomena into structures and systems \\-- Means of shaping and framing research questions}
	
	\end{itemize}

\end{frame}


\begin{frame}{Six questions to articulate the discussions}

    \begin{enumerate}

\item<1-> Why should you use cognitive architectures - how would they benefit your
    research as a theoretical framework, a tool and/or a methodology?

\item<2-> Should cognitive architectures for social interaction be inspired and/or
    limited by models of human cognition?

\item<3-> What are the functional requirements for a cognitive architecture to
    support social interaction?

\item<4-> How the requirements for social interaction would inform your choice of
    the fundamental computational structures of the architecture (e.g. symbolic,
    sub-symbolic, hybrid, ...)?

\item<5-> What is the primary outstanding challenge in developing and/or applying
    cognitive architectures to social HRI systems?

\item<6-> Can you devise a social interaction scenario that current cognitive
    architectures would likely fail, and why?

    \end{enumerate}
\end{frame}

\begin{frame}{Eight 15min presentations}
\footnotesize
    \begin{itemize}

\item Paul Baxter -- {\Medium Memory-Centred Cognitive Architectures for Robots
    Interacting Socially with Humans}


\item Sandra Devin, Grégoire Milliez, Michelangelo Fiore, Aurélie Clodic and
    Rachid Alami -- {\Medium Some Essential Skills and their Combination in an
    Architecture for a Cognitive and Interactive Robot}

\item Nick Depalma and Cynthia Breazeal -- {\Medium NIMBUS: A Hybrid
        Cloud-Crowd Realtime Architecture for Visual Learning in Interactive
    Domains}


\item Gabriel Ferrer -- {\Medium Associative Memories and Human-Robot Social
    Interaction}


\item Alexis Jacq, Wafa Johal, Pierre Dillenbourg and Ana Paiva -- {\Medium
    Cognitive Architecture for Mutual Modelling}

\item J. Ignacio Serrano and M. Dolores Del Castillo -- {\Medium The Three
    Must-Get-Theres.  Technical, Epistemological and Ethical Concerns about
    Cognitive Architectures for Social Interaction}

\item Vasanth Sarathy, Jason Wilson, Thomas Arnold and Matthias Scheutz --
    {\Medium Enabling Basic Normative HRI in a Cognitive Robotic Architectures}

\item Liz Sonenberg, Tim Miller, Adrian Pearce, Paolo Felli, Christian Muise and
    Frank Dignum -- {\Medium Social Planning for Social HRI}
    \end{itemize}
\end{frame}

\begin{frame}{Desired Outcomes}
    
    \begin{itemize}
    
    \item To bringing concepts and methods of Cognitive Architectures to the attention of the HRI community \\ {\footnotesize -- Specifically highlight the necessity for \emph{social} HRI}
    
    \item To facilitate discussion between participants from different backgrounds by providing a common frame of reference \\ {\footnotesize -- The six questions}
    
    \item To generate a reference point for the community: organise a Special Issue (open call) \\ {\footnotesize -- Interest from ``IEEE Trans. on Cognitive and Developmental Systems'' \\-- Answering the six questions will be part of the review criteria!}
    
    \end{itemize}
    
\end{frame}

\begin{frame}[plain]{}
    \centering
    {\Medium Plenty of time for discussions!}
\end{frame}

\begin{frame}{Proposed structure}
    \centering
    {\Medium Six discussions}\par
    \uncover<2->{Paul/Séverin give a short (10min) introduction to the question}\par
    \uncover<3->{35min of open discussion...}\par
    \uncover<4->{...interleaved with one or two presentations}\par

\end{frame}


\begin{frame}{Programme}

    \begin{table}[]
        \begin{tabularx}{\linewidth}{lp{5.5cm}l}
            \toprule
            {\Medium 9:00 -- 9:15} & Introduction \\
            {\Medium 9:15 -- 10:30} & {\Medium\doublequoted{Cognitive Architectures?}} &
                                      P1, P2 \\
            \midrule
            Coffee break & \\
            \midrule
            {\Medium 11:00 -- 12:00} & {\Medium\doublequoted{Inspiration from humans?}} & P3 \\
            \midrule
            Lunch & \\
            \midrule
            {\Medium 13:00 -- 14:00} & {\Medium\doublequoted{Functional Requirements}} & P4 \\
            {\Medium 14:00 -- 15:00} & {\Medium\doublequoted{Computational Structures}} & P5 \\
            \midrule
            Coffee break & \\
            \midrule
            {\Medium 15:30 -- 16:30} & {\Medium\doublequoted{Challenges of social HRI}} & P6 \\
            {\Medium 16:30 -- 17:45} & {\Medium\doublequoted{Social interaction scenarios}} & P7, P8 \\
            \midrule
            {\Medium $\rightarrow$ 18:00} & Wrap-up \\
            \bottomrule
        \end{tabularx}
        \label{tab:options}
    \end{table}

\end{frame}

\end{document}







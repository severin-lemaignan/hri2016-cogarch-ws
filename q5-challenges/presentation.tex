%!TEX program = xelatex

\documentclass[compress]{beamer}
%--------------------------------------------------------------------------
% Common packages
%--------------------------------------------------------------------------
\usepackage[english]{babel}
\usepackage{pgfpages} % required for notes on second screen
\usepackage{graphicx}

\usepackage{multicol}

\usepackage{tabularx,ragged2e}
\usepackage{booktabs}


%--------------------------------------------------------------------------
% Load theme
%--------------------------------------------------------------------------
\usetheme{hri}

\usepackage{dtklogos} % must be loaded after theme
\usepackage{tikz}
\usetikzlibrary{mindmap,backgrounds,positioning}

\graphicspath{{figs/}}
\setbeamercolor{refToContribCol}{bg=hriSec1Comp,fg=white}
\newcommand{\refToContrib}[1]{%
    \begin{beamercolorbox}[wd=\linewidth,ht=2ex,dp=0.7ex]{refToContribCol}%
    \hspace{0.5em}$\hookrightarrow$ #1%
    \end{beamercolorbox}%
}%

%--------------------------------------------------------------------------
% General presentation settings
%--------------------------------------------------------------------------
\title{Challenges of Social HRI}
\subtitle{Q5 -- What is the primary outstanding challenge in developing and/or
applying cognitive architectures to social HRI systems?}
\date{7th March 2016\\ {\tiny \url{https://sites.google.com/site/cogarch4socialhri2016/}}}
\author{}
\institute{\includegraphics[height=15mm]{plymouth-logo}}
%\\Centre for Robotics \& Neural
%Systems\\{\Medium Plymouth University}}

%--------------------------------------------------------------------------
% Notes settings
%--------------------------------------------------------------------------
%\setbeameroption{show notes on second screen}

\begin{document}
%--------------------------------------------------------------------------
% Titlepage
%--------------------------------------------------------------------------

\maketitle

%\imageframe[Children playing with\\the Ranger robot]{photo-fullscreen.jpg}
%\imageframe{photo-fullscreen.jpg}


\begin{frame}{Major Challenges}

    \begin{multicols}{2}
    
        \resizebox{0.7\columnwidth}{!}{%
        
            \begin{tikzpicture}[
                    >=latex,
                every edge/.style={<-, draw, very thick}]

            \path[small mindmap, 
                level 1 concept/.append style={sibling angle=360/6}, 
                level 2 concept/.append style={sibling angle=60}, 
            concept color=hriSec1,text=white]
            node[concept] {Human Cognition?}
                [clockwise from=60]
                child[concept color=hriSec3Dark,text=white] { node[concept]{1. Sensory Limits} }
                child[concept color=hriSec2Dark,text=white] { node[concept]{2. Long-term} }
                child[concept color=hriSec2CompDark,text=white] { node[concept] {3. Human-like?}};


        \end{tikzpicture}
    }
    
    \vspace{1cm}
           
    \refToContrib{All contributions}
    
    \vfill
    \columnbreak

    \uncover<2->{ {\Medium 1. Sensory limitations}

    {\scriptsize -- Automatically characterising fine-grained human behaviour \\-- The cognitive interpretation thereof} }

    \uncover<3->{ {\Medium 2. Achieving long-term interaction}

    {\scriptsize -- Stability of adaptive/learning mechanisms \\-- Metrics of evaluation (subjective?)} }

    \uncover<4->{ {\Medium 3. Should the control and behaviour be explicitly human-like?}

    {\scriptsize -- To what extent should CogArch for social interaction be constrained by human cognition in social behaviour? \\-- c.f. discussion of Q2...} }

    \end{multicols}
    
    
\end{frame}



\begin{frame}{}

    \centering

    \begin{exampleblock}{The Major Challenge seems to be handling the real world}
        With all the intricacies and complexities that this implies
    \end{exampleblock}
    \vspace{2em}
    Such a broad characterisation is not so useful, hence the following discussion points...


\end{frame}



\begin{frame}{Discussion points}
    
    -- Predictability vs Novelty
	    
    -- Handling Ambiguity: in terms of context, sensory mis-interpretation and the behavioural consequences thereof (asking for help, probabilistic reasoning, etc)
    \refToContrib{Papers by Alexis, Gabriel, ...}
    
	-- Measuring success in CogArch for Social Interaction: will the only metrics be the perceived quality of the interaction by the interacting humans, or can there be objective metrics?
	\refToContrib{Papers Nick, ...} 
    
    \vspace{0.5cm}

    35 minutes of open discussion, plus one presentation

\end{frame}


%\begin{frame}{Bibliography}
%
%	\begin{thebibliography}{10}
%
%    \beamertemplatearticlebibitems
%    \bibitem{lemaignan2015mutual}
%    S. Lemaignan, P. Dillenbourg
%    \newblock \doublequoted{Mutual Modelling: Inspiration for the Next Steps}
%    \newblock 2015
%
%    \beamertemplatearticlebibitems
%    \bibitem{tenorth2010understanding}
%    M. Tenorth, D. Nyga, M. Beetz
%    \newblock \doublequoted{Understanding and Executing Instructions for Everyday Manipulation Tasks
%    from the World Wide Web}
%    \newblock 2010
%
%    \beamertemplatearticlebibitems
%    \bibitem{mavridis2015review}
%    N. Mavridis
%    \newblock \doublequoted{A review of verbal and non-verbal human–robot interactive communication}
%    \newblock 2015
%
%    \beamertemplatearticlebibitems
%    \bibitem{kruijff2010situated}
%    G.-J. M. Kruijff \emph{et al.}
%    \newblock \doublequoted{Situated dialogue processing for human-robot interaction}
%    \newblock 2010
%
%	\end{thebibliography}
%  
%\end{frame}

\end{document}







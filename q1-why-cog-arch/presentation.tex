%!TEX program = xelatex

\documentclass[compress]{beamer}
%--------------------------------------------------------------------------
% Common packages
%--------------------------------------------------------------------------
\usepackage[english]{babel}
\usepackage{pgfpages} % required for notes on second screen
\usepackage{graphicx}

\usepackage{multicol}
\usepackage{url}

\usepackage{tabularx,ragged2e}
\usepackage{booktabs}


%--------------------------------------------------------------------------
% Load theme
%--------------------------------------------------------------------------
\usetheme[basicfont]{hri}

\usepackage{dtklogos} % must be loaded after theme
\usepackage{tikz}
\usetikzlibrary{mindmap,backgrounds,positioning}

\graphicspath{{figs/}}
\setbeamercolor{refToContribCol}{bg=hriSec1Comp,fg=white}
\newcommand{\refToContrib}[1]{%
    \begin{beamercolorbox}[wd=\linewidth,ht=2ex,dp=0.7ex]{refToContribCol}%
    \hspace{0.5em}$\hookrightarrow$ #1%
    \end{beamercolorbox}%
}%

%--------------------------------------------------------------------------
% General presentation settings
%--------------------------------------------------------------------------
\title{CogArch4socialHRI}
\subtitle{Q1 -- Why should you use cognitive architectures - how would they
benefit your research as a theoretical framework, a tool and/or a methodology?}
\date{7th March 2016\\ {\tiny \url{https://sites.google.com/site/cogarch4socialhri2016/}}}
\author{}
\institute{\includegraphics[height=15mm]{plymouth-logo}}
%\\Centre for Robotics \& Neural
%Systems\\{\Medium Plymouth University}}

%--------------------------------------------------------------------------
% Notes settings
%--------------------------------------------------------------------------
%\setbeameroption{show notes on second screen}

\begin{document}
%--------------------------------------------------------------------------
% Titlepage
%--------------------------------------------------------------------------

\maketitle

%\imageframe[Children playing with\\the Ranger robot]{photo-fullscreen.jpg}
%\imageframe{photo-fullscreen.jpg}


\begin{frame}{Three Interpretations of CogArch}

    \begin{multicols}{2}
        \resizebox{0.7\columnwidth}{!}{%
            \begin{tikzpicture}[
                    >=latex,
                every edge/.style={<-, draw, very thick}]

            \path[small mindmap, 
                level 1 concept/.append style={sibling angle=360/6}, 
                level 2 concept/.append style={sibling angle=60}, 
            concept color=hriSec1,text=white]
            node[concept] {Cognitive Architectures}
                [clockwise from=60]
                child[concept color=hriSec3Dark,text=white] { node[concept]{Model} }
                child[concept color=hriSec2Dark,text=white] { node[concept]{Integration} }
                child[concept color=hriSec2CompDark,text=white] { node[concept] {Methodology}};


        \end{tikzpicture}
    }
    \vfill
    \columnbreak

    {\Medium Model of Human Cognition}

    Recognition {\tiny \Medium facial, non-verbal acoustic, verbal features / sentient analysis}\\
    Interpretation {\tiny \Medium empathy,...}

    {\Medium Technical Integration}

    Visual attention\\
    Joint attention

    {\Medium Architecture as Methodology}

    User modelling\\
    (perceptual) theory of mind~\cite{lemaignan2015mutual}

    \refToContrib{Someone here...}

    \end{multicols}
\end{frame}



\begin{frame}{Bringing the strands together}
    

    \begin{itemize}
        \item ...
        \item ...
        \item ...
    \end{itemize}

    \refToContrib{Someone (15 min)}
    \refToContrib{Someone (15 min)}

\end{frame}


\begin{frame}{Bibliography}
\begin{thebibliography}{10}

    \beamertemplatearticlebibitems
    \bibitem{lemaignan2015mutual}
    S. Lemaignan, P. Dillenbourg
    \newblock \doublequoted{Mutual Modelling: Inspiration for the Next Steps}
    \newblock 2015

    \beamertemplatearticlebibitems
    \bibitem{tenorth2010understanding}
    M. Tenorth, D. Nyga, M. Beetz
    \newblock \doublequoted{Understanding and Executing Instructions for Everyday Manipulation Tasks
    from the World Wide Web}
    \newblock 2010

    \beamertemplatearticlebibitems
    \bibitem{mavridis2015review}
    N. Mavridis
    \newblock \doublequoted{A review of verbal and non-verbal human–robot interactive communication}
    \newblock 2015



    \beamertemplatearticlebibitems
    \bibitem{kruijff2010situated}
    G.-J. M. Kruijff \emph{et al.}
    \newblock \doublequoted{Situated dialogue processing for human-robot interaction}
    \newblock 2010



  \end{thebibliography}
\end{frame}

\end{document}







%!TEX program = xelatex

\documentclass[compress]{beamer}
%--------------------------------------------------------------------------
% Common packages
%--------------------------------------------------------------------------
\usepackage[english]{babel}
\usepackage{pgfpages} % required for notes on second screen
\usepackage{graphicx}

\usepackage{multicol}
\usepackage{url}

\usepackage{tabularx,ragged2e}
\usepackage{booktabs}


%--------------------------------------------------------------------------
% Load theme
%--------------------------------------------------------------------------
\usetheme{hri}

\usepackage{dtklogos} % must be loaded after theme
\usepackage{tikz}
\usetikzlibrary{calc,mindmap,backgrounds,positioning}

\graphicspath{{figs/}}
\setbeamercolor{refToContribCol}{bg=hriSec1Comp,fg=white}
\newcommand{\refToContrib}[1]{%
    \begin{beamercolorbox}[wd=\linewidth,ht=2ex,dp=0.7ex]{refToContribCol}%
    \hspace{0.5em}$\hookrightarrow$ #1%
    \end{beamercolorbox}%
}%

%--------------------------------------------------------------------------
% General presentation settings
%--------------------------------------------------------------------------
\title{Cognitive Architectures?}
\subtitle{Q1 -- Why should you use cognitive architectures - how would they
benefit your research as a theoretical framework, a tool and/or a methodology?}
\date{7th March 2016\\ {\tiny \url{https://sites.google.com/site/cogarch4socialhri2016/}}}
\author{}
\institute{\includegraphics[height=15mm]{plymouth-logo}}
%\\Centre for Robotics \& Neural
%Systems\\{\Medium Plymouth University}}

%--------------------------------------------------------------------------
% Notes settings
%--------------------------------------------------------------------------
%\setbeameroption{show notes on second screen}

\begin{document}
%--------------------------------------------------------------------------
% Titlepage
%--------------------------------------------------------------------------

\maketitle

%\imageframe[Children playing with\\the Ranger robot]{photo-fullscreen.jpg}
%\imageframe{photo-fullscreen.jpg}


\begin{frame}{Three Interpretations of CogArch}

    \begin{multicols}{2}
        \resizebox{0.7\columnwidth}{!}{%
            \begin{tikzpicture}[
                    >=latex,
                every edge/.style={<-, draw, very thick}]

            \path[small mindmap, 
                level 1 concept/.append style={sibling angle=360/6}, 
                level 2 concept/.append style={sibling angle=60}, 
            concept color=hriSec1,text=white]
            node[concept] {Cognitive Architectures}
                [clockwise from=60]
                child[concept color=hriSec3Dark,text=white] { node[concept]{1. Model} }
                child[concept color=hriSec2Dark,text=white] { node[concept]{2. Integration} }
                child[concept color=hriSec2CompDark,text=white] { node[concept] {3. Methodology}};


        \end{tikzpicture}
    }
    \vfill
    \columnbreak

    {\Medium 1. Models of Human Cognition}

    {\scriptsize -- Modelling (aspects of) human cognition \\-- Subsequent application to robots}

    {\Medium 2. Technical Integration}

    {\scriptsize -- Define required functionality of robots \\-- Implement algorithms (etc) necessary}

    {\Medium 3. CogArch as Methodology}

    {\scriptsize -- Formalising assumptions \\-- Integrating knowledge from multiple disciplines \\-- Iteratively updating architecture}

    %\refToContrib{Someone here...}

    \end{multicols}
\end{frame}

\begin{frame}{1. Models of Human Cognition}    

    \begin{itemize}
        \item Model some human cognitive phenomenon, or preferably set of phenomena (typically based on human behavioural data)
        \item Derive operating principles/algorithms that fulfil the requirements of the human behaviour
        \item Verify/validate resulting model by fitting to existing human data, or making predictions of human behaviour
        \item Apply model to robotics system
    \end{itemize}
    
    Multiple examples in the literature of (aspects of) this process, at least starting from an existing cognitive architecture
    
    \refToContrib{Most contributions fit here in some way}

\end{frame}


\begin{frame}{1. Models of Human Cognition}    

    \includegraphics[height=60mm]{cogarch-cognitive-integration}

\end{frame}


\begin{frame}{Architectures to model human cognition}

\resizebox{!}{0.7\paperheight}{%
\tikzset{subpart/.style={draw, font=\scriptsize, fill opacity=0.5, text opacity=1, fill=white!50}}
\begin{tikzpicture}[
    >=latex,
    node distance=1.5,
    every edge/.style={draw, very thick},
    skill/.style={draw, rounded corners, align=center, inner sep=5pt, fill=black!20},
    stmt/.style={align=center, font=\Medium},
    label/.style={midway, align=center, font=\scriptsize, fill=white}]

    \node at (0,0)[skill, fill=hriSec2!50] (a1) {Shared Plan Elaboration};

    \node [skill, fill=hriSec2!50,above=of a1] (a2) {Intention Prediction};
    \node [skill, fill=hriSec2!50,left=of a1] (a3) {Mental State Management};
    \node [skill, fill=hriSec2!50,right=of a1] (a4) {Communication for\\Joint Action};
    \node [skill, fill=hriSec2!50,below=of a1] (a5) {Shared Plan Achievement};
    \node [skill, fill=hriSec2!50,left=of a5] (a6) {Situation Assessment};


    \node [skill, fill=hriSec3!50,below left=of a5,anchor=north] (a7) {Action Achievement};
    \node [skill, fill=hriSec3!50,below right=of a5,anchor=north] (a8) {Human Action Monitoring};
  
    \node[below=3.7 of a5] (a14) {Human-aware geometric and task planners, real-time controllers, sensors...};

  \coordinate[below=3 of a6] (a9);

  \node[rotate=90,left=0.7 of a3.west] (distal) {\Medium\large DISTAL};
  \node[rotate=90] at (distal |- a7.south) {\Medium\large PROXIMAL};
  \node[rotate=90] at (distal |- a14) {\Medium\large MOTOR};

  \coordinate (a11) at (a9 -| distal.north);
  \coordinate (a12) at (a9 -| a4.east);
  \draw[dotted, thick] (a11) -- (a12);


  \coordinate (a13) at ($(a5)!0.5!(a7)$);
  \draw[dotted, thick] (a13 -| distal.north) -- (a13 -| a4.east);


  %%% Relations between components
  \path (a2) edge [->] node[label] {goal to execute} (a1);
  \path (a1) edge [->] node[label] {plan} (a5);
  \path (a2) edge [<-] node[label] {goal (order)} (a4);

  \path (a3) edge [<->, bend left=40, looseness=1.2] node[label,pos=0.1] {mental state information} (a4);
  \path (a3) edge [<->] (a2);
  \path (a3) edge [<->] (a1);
  \path (a3) edge [<->] (a5);

  \path (a6) edge [->] node[label] {conceptual\\world state} (a3);

  \path (a5) edge [<->] node[label] {coordination} (a4);

  \path (a5) edge [<->] node[label,right=0.5] {anchoring of actions} (a7);
  \path (a5) edge [<->] (a8);

  \path (a9) edge [->] node[label] {sensors} (a6);

  \coordinate (a10) at (a9 -| a7);
  \path (a10) edge [<->] node[label] {planning and control} (a7);

  \coordinate (a10) at (a9 -| a8);
  \path (a10) edge [<->] node[label] {sensors} (a8);

  \path (a7) edge [<->] node[label] {coordination} (a8);
 
\end{tikzpicture}
}

\begin{flushright}
\scriptsize \refToContrib{From Sandra's contribution (Paper 1)}
\end{flushright}

\end{frame}


\begin{frame}{2. Technical Integration}    

    \begin{itemize}
        \item Start with the application domain/problem to be solved
        \item Define set of algorithms required to fulfil task
        \item Implement architecture on robot
        \item Run experiment
    \end{itemize}
    
	A few examples in the literature of (aspects of) this process.
	\refToContrib{Liz (social planning), Nick (percept learning), ...}

\end{frame}

\begin{frame}{2. Technical Integration}    

    \includegraphics[height=60mm]{cogarch-technical-integration}

\end{frame}


\begin{frame}{3. CogArch as Methodology}    

    \begin{itemize}
        \item Human-derived evidence (from perhaps multiple domains)
        \item Define set of principles and constraints
        \item Implement architecture on robot; generate predictions; run experiment(s)
        \item Feed back results into principles and constraints; repeat
    \end{itemize}
    
    This process not typically explicit in the literature, but rather implicit in the development of cognitive (model) architectures in psychology/CogSci for example. Also includes those approaches that emphasise fundamental mechanisms rather than human-level functionality.    
    \refToContrib{Gabriel (SOMs), Paul (associative memory), ...}
    
\end{frame}


\begin{frame}{3. CogArch as Methodology}    

    \includegraphics[height=60mm]{cogarch-methodology}

\end{frame}


\imageframe[white]{Islands of desired Cognitive Functionality}{figs/islands1}
\imageframe[white]{Linking islands together: the technical integration approach}{figs/islands2}
\imageframe[white]{Exploring common underlying principles of the islands}{figs/islands3}
\imageframe[white]{Constructing cognitive architectures from these principles}{figs/islands4}


\begin{frame}{Discussion: why use CogArch for social HRI?}
    

    \begin{itemize}
        \item To what extent are `shallow' models of robot cognitive behaviour (e.g. reactivity) sufficient to lead to complex social behaviour?

        \item Is the application of a human-inspired complete model of cognition simply over-the-top for the desired functionality?
        \item The use of Cognitive Architectures to integrate knowledge (empirical data) from different disciplines into sHRI...
        \item How robots tranform our existing understanding of CogArch for
            social interaction?
    \end{itemize}
    
	\vspace{0.5cm}	

    35 minutes of open discussion + 2 presentations

\end{frame}


%\begin{frame}{Bibliography}
%
%	\begin{thebibliography}{10}
%
%    \beamertemplatearticlebibitems
%    \bibitem{lemaignan2015mutual}
%    S. Lemaignan, P. Dillenbourg
%    \newblock \doublequoted{Mutual Modelling: Inspiration for the Next Steps}
%    \newblock 2015
%
%    \beamertemplatearticlebibitems
%    \bibitem{tenorth2010understanding}
%    M. Tenorth, D. Nyga, M. Beetz
%    \newblock \doublequoted{Understanding and Executing Instructions for Everyday Manipulation Tasks
%    from the World Wide Web}
%    \newblock 2010
%
%    \beamertemplatearticlebibitems
%    \bibitem{mavridis2015review}
%    N. Mavridis
%    \newblock \doublequoted{A review of verbal and non-verbal human–robot interactive communication}
%    \newblock 2015
%
%    \beamertemplatearticlebibitems
%    \bibitem{kruijff2010situated}
%    G.-J. M. Kruijff \emph{et al.}
%    \newblock \doublequoted{Situated dialogue processing for human-robot interaction}
%    \newblock 2010
%
%  	\end{thebibliography}
%  
%\end{frame}

\end{document}






